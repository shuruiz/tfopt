
% Default to the notebook output style

    


% Inherit from the specified cell style.




    
\documentclass[11pt]{article}

    
    
    \usepackage[T1]{fontenc}
    % Nicer default font (+ math font) than Computer Modern for most use cases
    \usepackage{mathpazo}

    % Basic figure setup, for now with no caption control since it's done
    % automatically by Pandoc (which extracts ![](path) syntax from Markdown).
    \usepackage{graphicx}
    % We will generate all images so they have a width \maxwidth. This means
    % that they will get their normal width if they fit onto the page, but
    % are scaled down if they would overflow the margins.
    \makeatletter
    \def\maxwidth{\ifdim\Gin@nat@width>\linewidth\linewidth
    \else\Gin@nat@width\fi}
    \makeatother
    \let\Oldincludegraphics\includegraphics
    % Set max figure width to be 80% of text width, for now hardcoded.
    \renewcommand{\includegraphics}[1]{\Oldincludegraphics[width=.8\maxwidth]{#1}}
    % Ensure that by default, figures have no caption (until we provide a
    % proper Figure object with a Caption API and a way to capture that
    % in the conversion process - todo).
    \usepackage{caption}
    \DeclareCaptionLabelFormat{nolabel}{}
    \captionsetup{labelformat=nolabel}

    \usepackage{adjustbox} % Used to constrain images to a maximum size 
    \usepackage{xcolor} % Allow colors to be defined
    \usepackage{enumerate} % Needed for markdown enumerations to work
    \usepackage{geometry} % Used to adjust the document margins
    \usepackage{amsmath} % Equations
    \usepackage{amssymb} % Equations
    \usepackage{textcomp} % defines textquotesingle
    % Hack from http://tex.stackexchange.com/a/47451/13684:
    \AtBeginDocument{%
        \def\PYZsq{\textquotesingle}% Upright quotes in Pygmentized code
    }
    \usepackage{upquote} % Upright quotes for verbatim code
    \usepackage{eurosym} % defines \euro
    \usepackage[mathletters]{ucs} % Extended unicode (utf-8) support
    \usepackage[utf8x]{inputenc} % Allow utf-8 characters in the tex document
    \usepackage{fancyvrb} % verbatim replacement that allows latex
    \usepackage{grffile} % extends the file name processing of package graphics 
                         % to support a larger range 
    % The hyperref package gives us a pdf with properly built
    % internal navigation ('pdf bookmarks' for the table of contents,
    % internal cross-reference links, web links for URLs, etc.)
    \usepackage{hyperref}
    \usepackage{longtable} % longtable support required by pandoc >1.10
    \usepackage{booktabs}  % table support for pandoc > 1.12.2
    \usepackage[inline]{enumitem} % IRkernel/repr support (it uses the enumerate* environment)
    \usepackage[normalem]{ulem} % ulem is needed to support strikethroughs (\sout)
                                % normalem makes italics be italics, not underlines
    

    
    
    % Colors for the hyperref package
    \definecolor{urlcolor}{rgb}{0,.145,.698}
    \definecolor{linkcolor}{rgb}{.71,0.21,0.01}
    \definecolor{citecolor}{rgb}{.12,.54,.11}

    % ANSI colors
    \definecolor{ansi-black}{HTML}{3E424D}
    \definecolor{ansi-black-intense}{HTML}{282C36}
    \definecolor{ansi-red}{HTML}{E75C58}
    \definecolor{ansi-red-intense}{HTML}{B22B31}
    \definecolor{ansi-green}{HTML}{00A250}
    \definecolor{ansi-green-intense}{HTML}{007427}
    \definecolor{ansi-yellow}{HTML}{DDB62B}
    \definecolor{ansi-yellow-intense}{HTML}{B27D12}
    \definecolor{ansi-blue}{HTML}{208FFB}
    \definecolor{ansi-blue-intense}{HTML}{0065CA}
    \definecolor{ansi-magenta}{HTML}{D160C4}
    \definecolor{ansi-magenta-intense}{HTML}{A03196}
    \definecolor{ansi-cyan}{HTML}{60C6C8}
    \definecolor{ansi-cyan-intense}{HTML}{258F8F}
    \definecolor{ansi-white}{HTML}{C5C1B4}
    \definecolor{ansi-white-intense}{HTML}{A1A6B2}

    % commands and environments needed by pandoc snippets
    % extracted from the output of `pandoc -s`
    \providecommand{\tightlist}{%
      \setlength{\itemsep}{0pt}\setlength{\parskip}{0pt}}
    \DefineVerbatimEnvironment{Highlighting}{Verbatim}{commandchars=\\\{\}}
    % Add ',fontsize=\small' for more characters per line
    \newenvironment{Shaded}{}{}
    \newcommand{\KeywordTok}[1]{\textcolor[rgb]{0.00,0.44,0.13}{\textbf{{#1}}}}
    \newcommand{\DataTypeTok}[1]{\textcolor[rgb]{0.56,0.13,0.00}{{#1}}}
    \newcommand{\DecValTok}[1]{\textcolor[rgb]{0.25,0.63,0.44}{{#1}}}
    \newcommand{\BaseNTok}[1]{\textcolor[rgb]{0.25,0.63,0.44}{{#1}}}
    \newcommand{\FloatTok}[1]{\textcolor[rgb]{0.25,0.63,0.44}{{#1}}}
    \newcommand{\CharTok}[1]{\textcolor[rgb]{0.25,0.44,0.63}{{#1}}}
    \newcommand{\StringTok}[1]{\textcolor[rgb]{0.25,0.44,0.63}{{#1}}}
    \newcommand{\CommentTok}[1]{\textcolor[rgb]{0.38,0.63,0.69}{\textit{{#1}}}}
    \newcommand{\OtherTok}[1]{\textcolor[rgb]{0.00,0.44,0.13}{{#1}}}
    \newcommand{\AlertTok}[1]{\textcolor[rgb]{1.00,0.00,0.00}{\textbf{{#1}}}}
    \newcommand{\FunctionTok}[1]{\textcolor[rgb]{0.02,0.16,0.49}{{#1}}}
    \newcommand{\RegionMarkerTok}[1]{{#1}}
    \newcommand{\ErrorTok}[1]{\textcolor[rgb]{1.00,0.00,0.00}{\textbf{{#1}}}}
    \newcommand{\NormalTok}[1]{{#1}}
    
    % Additional commands for more recent versions of Pandoc
    \newcommand{\ConstantTok}[1]{\textcolor[rgb]{0.53,0.00,0.00}{{#1}}}
    \newcommand{\SpecialCharTok}[1]{\textcolor[rgb]{0.25,0.44,0.63}{{#1}}}
    \newcommand{\VerbatimStringTok}[1]{\textcolor[rgb]{0.25,0.44,0.63}{{#1}}}
    \newcommand{\SpecialStringTok}[1]{\textcolor[rgb]{0.73,0.40,0.53}{{#1}}}
    \newcommand{\ImportTok}[1]{{#1}}
    \newcommand{\DocumentationTok}[1]{\textcolor[rgb]{0.73,0.13,0.13}{\textit{{#1}}}}
    \newcommand{\AnnotationTok}[1]{\textcolor[rgb]{0.38,0.63,0.69}{\textbf{\textit{{#1}}}}}
    \newcommand{\CommentVarTok}[1]{\textcolor[rgb]{0.38,0.63,0.69}{\textbf{\textit{{#1}}}}}
    \newcommand{\VariableTok}[1]{\textcolor[rgb]{0.10,0.09,0.49}{{#1}}}
    \newcommand{\ControlFlowTok}[1]{\textcolor[rgb]{0.00,0.44,0.13}{\textbf{{#1}}}}
    \newcommand{\OperatorTok}[1]{\textcolor[rgb]{0.40,0.40,0.40}{{#1}}}
    \newcommand{\BuiltInTok}[1]{{#1}}
    \newcommand{\ExtensionTok}[1]{{#1}}
    \newcommand{\PreprocessorTok}[1]{\textcolor[rgb]{0.74,0.48,0.00}{{#1}}}
    \newcommand{\AttributeTok}[1]{\textcolor[rgb]{0.49,0.56,0.16}{{#1}}}
    \newcommand{\InformationTok}[1]{\textcolor[rgb]{0.38,0.63,0.69}{\textbf{\textit{{#1}}}}}
    \newcommand{\WarningTok}[1]{\textcolor[rgb]{0.38,0.63,0.69}{\textbf{\textit{{#1}}}}}
    
    
    % Define a nice break command that doesn't care if a line doesn't already
    % exist.
    \def\br{\hspace*{\fill} \\* }
    % Math Jax compatability definitions
    \def\gt{>}
    \def\lt{<}
    % Document parameters
    \title{Hill climb and annealing using scipy optimize}
    
    
    

    % Pygments definitions
    
\makeatletter
\def\PY@reset{\let\PY@it=\relax \let\PY@bf=\relax%
    \let\PY@ul=\relax \let\PY@tc=\relax%
    \let\PY@bc=\relax \let\PY@ff=\relax}
\def\PY@tok#1{\csname PY@tok@#1\endcsname}
\def\PY@toks#1+{\ifx\relax#1\empty\else%
    \PY@tok{#1}\expandafter\PY@toks\fi}
\def\PY@do#1{\PY@bc{\PY@tc{\PY@ul{%
    \PY@it{\PY@bf{\PY@ff{#1}}}}}}}
\def\PY#1#2{\PY@reset\PY@toks#1+\relax+\PY@do{#2}}

\expandafter\def\csname PY@tok@w\endcsname{\def\PY@tc##1{\textcolor[rgb]{0.73,0.73,0.73}{##1}}}
\expandafter\def\csname PY@tok@c\endcsname{\let\PY@it=\textit\def\PY@tc##1{\textcolor[rgb]{0.25,0.50,0.50}{##1}}}
\expandafter\def\csname PY@tok@cp\endcsname{\def\PY@tc##1{\textcolor[rgb]{0.74,0.48,0.00}{##1}}}
\expandafter\def\csname PY@tok@k\endcsname{\let\PY@bf=\textbf\def\PY@tc##1{\textcolor[rgb]{0.00,0.50,0.00}{##1}}}
\expandafter\def\csname PY@tok@kp\endcsname{\def\PY@tc##1{\textcolor[rgb]{0.00,0.50,0.00}{##1}}}
\expandafter\def\csname PY@tok@kt\endcsname{\def\PY@tc##1{\textcolor[rgb]{0.69,0.00,0.25}{##1}}}
\expandafter\def\csname PY@tok@o\endcsname{\def\PY@tc##1{\textcolor[rgb]{0.40,0.40,0.40}{##1}}}
\expandafter\def\csname PY@tok@ow\endcsname{\let\PY@bf=\textbf\def\PY@tc##1{\textcolor[rgb]{0.67,0.13,1.00}{##1}}}
\expandafter\def\csname PY@tok@nb\endcsname{\def\PY@tc##1{\textcolor[rgb]{0.00,0.50,0.00}{##1}}}
\expandafter\def\csname PY@tok@nf\endcsname{\def\PY@tc##1{\textcolor[rgb]{0.00,0.00,1.00}{##1}}}
\expandafter\def\csname PY@tok@nc\endcsname{\let\PY@bf=\textbf\def\PY@tc##1{\textcolor[rgb]{0.00,0.00,1.00}{##1}}}
\expandafter\def\csname PY@tok@nn\endcsname{\let\PY@bf=\textbf\def\PY@tc##1{\textcolor[rgb]{0.00,0.00,1.00}{##1}}}
\expandafter\def\csname PY@tok@ne\endcsname{\let\PY@bf=\textbf\def\PY@tc##1{\textcolor[rgb]{0.82,0.25,0.23}{##1}}}
\expandafter\def\csname PY@tok@nv\endcsname{\def\PY@tc##1{\textcolor[rgb]{0.10,0.09,0.49}{##1}}}
\expandafter\def\csname PY@tok@no\endcsname{\def\PY@tc##1{\textcolor[rgb]{0.53,0.00,0.00}{##1}}}
\expandafter\def\csname PY@tok@nl\endcsname{\def\PY@tc##1{\textcolor[rgb]{0.63,0.63,0.00}{##1}}}
\expandafter\def\csname PY@tok@ni\endcsname{\let\PY@bf=\textbf\def\PY@tc##1{\textcolor[rgb]{0.60,0.60,0.60}{##1}}}
\expandafter\def\csname PY@tok@na\endcsname{\def\PY@tc##1{\textcolor[rgb]{0.49,0.56,0.16}{##1}}}
\expandafter\def\csname PY@tok@nt\endcsname{\let\PY@bf=\textbf\def\PY@tc##1{\textcolor[rgb]{0.00,0.50,0.00}{##1}}}
\expandafter\def\csname PY@tok@nd\endcsname{\def\PY@tc##1{\textcolor[rgb]{0.67,0.13,1.00}{##1}}}
\expandafter\def\csname PY@tok@s\endcsname{\def\PY@tc##1{\textcolor[rgb]{0.73,0.13,0.13}{##1}}}
\expandafter\def\csname PY@tok@sd\endcsname{\let\PY@it=\textit\def\PY@tc##1{\textcolor[rgb]{0.73,0.13,0.13}{##1}}}
\expandafter\def\csname PY@tok@si\endcsname{\let\PY@bf=\textbf\def\PY@tc##1{\textcolor[rgb]{0.73,0.40,0.53}{##1}}}
\expandafter\def\csname PY@tok@se\endcsname{\let\PY@bf=\textbf\def\PY@tc##1{\textcolor[rgb]{0.73,0.40,0.13}{##1}}}
\expandafter\def\csname PY@tok@sr\endcsname{\def\PY@tc##1{\textcolor[rgb]{0.73,0.40,0.53}{##1}}}
\expandafter\def\csname PY@tok@ss\endcsname{\def\PY@tc##1{\textcolor[rgb]{0.10,0.09,0.49}{##1}}}
\expandafter\def\csname PY@tok@sx\endcsname{\def\PY@tc##1{\textcolor[rgb]{0.00,0.50,0.00}{##1}}}
\expandafter\def\csname PY@tok@m\endcsname{\def\PY@tc##1{\textcolor[rgb]{0.40,0.40,0.40}{##1}}}
\expandafter\def\csname PY@tok@gh\endcsname{\let\PY@bf=\textbf\def\PY@tc##1{\textcolor[rgb]{0.00,0.00,0.50}{##1}}}
\expandafter\def\csname PY@tok@gu\endcsname{\let\PY@bf=\textbf\def\PY@tc##1{\textcolor[rgb]{0.50,0.00,0.50}{##1}}}
\expandafter\def\csname PY@tok@gd\endcsname{\def\PY@tc##1{\textcolor[rgb]{0.63,0.00,0.00}{##1}}}
\expandafter\def\csname PY@tok@gi\endcsname{\def\PY@tc##1{\textcolor[rgb]{0.00,0.63,0.00}{##1}}}
\expandafter\def\csname PY@tok@gr\endcsname{\def\PY@tc##1{\textcolor[rgb]{1.00,0.00,0.00}{##1}}}
\expandafter\def\csname PY@tok@ge\endcsname{\let\PY@it=\textit}
\expandafter\def\csname PY@tok@gs\endcsname{\let\PY@bf=\textbf}
\expandafter\def\csname PY@tok@gp\endcsname{\let\PY@bf=\textbf\def\PY@tc##1{\textcolor[rgb]{0.00,0.00,0.50}{##1}}}
\expandafter\def\csname PY@tok@go\endcsname{\def\PY@tc##1{\textcolor[rgb]{0.53,0.53,0.53}{##1}}}
\expandafter\def\csname PY@tok@gt\endcsname{\def\PY@tc##1{\textcolor[rgb]{0.00,0.27,0.87}{##1}}}
\expandafter\def\csname PY@tok@err\endcsname{\def\PY@bc##1{\setlength{\fboxsep}{0pt}\fcolorbox[rgb]{1.00,0.00,0.00}{1,1,1}{\strut ##1}}}
\expandafter\def\csname PY@tok@kc\endcsname{\let\PY@bf=\textbf\def\PY@tc##1{\textcolor[rgb]{0.00,0.50,0.00}{##1}}}
\expandafter\def\csname PY@tok@kd\endcsname{\let\PY@bf=\textbf\def\PY@tc##1{\textcolor[rgb]{0.00,0.50,0.00}{##1}}}
\expandafter\def\csname PY@tok@kn\endcsname{\let\PY@bf=\textbf\def\PY@tc##1{\textcolor[rgb]{0.00,0.50,0.00}{##1}}}
\expandafter\def\csname PY@tok@kr\endcsname{\let\PY@bf=\textbf\def\PY@tc##1{\textcolor[rgb]{0.00,0.50,0.00}{##1}}}
\expandafter\def\csname PY@tok@bp\endcsname{\def\PY@tc##1{\textcolor[rgb]{0.00,0.50,0.00}{##1}}}
\expandafter\def\csname PY@tok@fm\endcsname{\def\PY@tc##1{\textcolor[rgb]{0.00,0.00,1.00}{##1}}}
\expandafter\def\csname PY@tok@vc\endcsname{\def\PY@tc##1{\textcolor[rgb]{0.10,0.09,0.49}{##1}}}
\expandafter\def\csname PY@tok@vg\endcsname{\def\PY@tc##1{\textcolor[rgb]{0.10,0.09,0.49}{##1}}}
\expandafter\def\csname PY@tok@vi\endcsname{\def\PY@tc##1{\textcolor[rgb]{0.10,0.09,0.49}{##1}}}
\expandafter\def\csname PY@tok@vm\endcsname{\def\PY@tc##1{\textcolor[rgb]{0.10,0.09,0.49}{##1}}}
\expandafter\def\csname PY@tok@sa\endcsname{\def\PY@tc##1{\textcolor[rgb]{0.73,0.13,0.13}{##1}}}
\expandafter\def\csname PY@tok@sb\endcsname{\def\PY@tc##1{\textcolor[rgb]{0.73,0.13,0.13}{##1}}}
\expandafter\def\csname PY@tok@sc\endcsname{\def\PY@tc##1{\textcolor[rgb]{0.73,0.13,0.13}{##1}}}
\expandafter\def\csname PY@tok@dl\endcsname{\def\PY@tc##1{\textcolor[rgb]{0.73,0.13,0.13}{##1}}}
\expandafter\def\csname PY@tok@s2\endcsname{\def\PY@tc##1{\textcolor[rgb]{0.73,0.13,0.13}{##1}}}
\expandafter\def\csname PY@tok@sh\endcsname{\def\PY@tc##1{\textcolor[rgb]{0.73,0.13,0.13}{##1}}}
\expandafter\def\csname PY@tok@s1\endcsname{\def\PY@tc##1{\textcolor[rgb]{0.73,0.13,0.13}{##1}}}
\expandafter\def\csname PY@tok@mb\endcsname{\def\PY@tc##1{\textcolor[rgb]{0.40,0.40,0.40}{##1}}}
\expandafter\def\csname PY@tok@mf\endcsname{\def\PY@tc##1{\textcolor[rgb]{0.40,0.40,0.40}{##1}}}
\expandafter\def\csname PY@tok@mh\endcsname{\def\PY@tc##1{\textcolor[rgb]{0.40,0.40,0.40}{##1}}}
\expandafter\def\csname PY@tok@mi\endcsname{\def\PY@tc##1{\textcolor[rgb]{0.40,0.40,0.40}{##1}}}
\expandafter\def\csname PY@tok@il\endcsname{\def\PY@tc##1{\textcolor[rgb]{0.40,0.40,0.40}{##1}}}
\expandafter\def\csname PY@tok@mo\endcsname{\def\PY@tc##1{\textcolor[rgb]{0.40,0.40,0.40}{##1}}}
\expandafter\def\csname PY@tok@ch\endcsname{\let\PY@it=\textit\def\PY@tc##1{\textcolor[rgb]{0.25,0.50,0.50}{##1}}}
\expandafter\def\csname PY@tok@cm\endcsname{\let\PY@it=\textit\def\PY@tc##1{\textcolor[rgb]{0.25,0.50,0.50}{##1}}}
\expandafter\def\csname PY@tok@cpf\endcsname{\let\PY@it=\textit\def\PY@tc##1{\textcolor[rgb]{0.25,0.50,0.50}{##1}}}
\expandafter\def\csname PY@tok@c1\endcsname{\let\PY@it=\textit\def\PY@tc##1{\textcolor[rgb]{0.25,0.50,0.50}{##1}}}
\expandafter\def\csname PY@tok@cs\endcsname{\let\PY@it=\textit\def\PY@tc##1{\textcolor[rgb]{0.25,0.50,0.50}{##1}}}

\def\PYZbs{\char`\\}
\def\PYZus{\char`\_}
\def\PYZob{\char`\{}
\def\PYZcb{\char`\}}
\def\PYZca{\char`\^}
\def\PYZam{\char`\&}
\def\PYZlt{\char`\<}
\def\PYZgt{\char`\>}
\def\PYZsh{\char`\#}
\def\PYZpc{\char`\%}
\def\PYZdl{\char`\$}
\def\PYZhy{\char`\-}
\def\PYZsq{\char`\'}
\def\PYZdq{\char`\"}
\def\PYZti{\char`\~}
% for compatibility with earlier versions
\def\PYZat{@}
\def\PYZlb{[}
\def\PYZrb{]}
\makeatother


    % Exact colors from NB
    \definecolor{incolor}{rgb}{0.0, 0.0, 0.5}
    \definecolor{outcolor}{rgb}{0.545, 0.0, 0.0}



    
    % Prevent overflowing lines due to hard-to-break entities
    \sloppy 
    % Setup hyperref package
    \hypersetup{
      breaklinks=true,  % so long urls are correctly broken across lines
      colorlinks=true,
      urlcolor=urlcolor,
      linkcolor=linkcolor,
      citecolor=citecolor,
      }
    % Slightly bigger margins than the latex defaults
    
    \geometry{verbose,tmargin=1in,bmargin=1in,lmargin=1in,rmargin=1in}
    
    

    \begin{document}
    
    
    \maketitle
    
    

    
    \section{\texorpdfstring{{Random Hill Climbing} and {Simulated
Annealing} algorithm
Demo}{Random Hill Climbing and Simulated Annealing algorithm Demo}}\label{random-hill-climbing-and-simulated-annealing-algorithm-demo}

\subsection{Tirthajyoti Sarkar, Sunnyvale, September
2018}\label{tirthajyoti-sarkar-sunnyvale-september-2018}

    \begin{Verbatim}[commandchars=\\\{\}]
{\color{incolor}In [{\color{incolor}8}]:} \PY{k+kn}{import} \PY{n+nn}{numpy} \PY{k}{as} \PY{n+nn}{np}
        \PY{k+kn}{import} \PY{n+nn}{pandas} \PY{k}{as} \PY{n+nn}{pd}
        \PY{k+kn}{import} \PY{n+nn}{seaborn} \PY{k}{as} \PY{n+nn}{sns}
        \PY{k+kn}{import} \PY{n+nn}{matplotlib}\PY{n+nn}{.}\PY{n+nn}{pyplot} \PY{k}{as} \PY{n+nn}{plt}
        \PY{k+kn}{from} \PY{n+nn}{scipy} \PY{k}{import} \PY{n}{optimize}
        \PY{k+kn}{from} \PY{n+nn}{sklearn}\PY{n+nn}{.}\PY{n+nn}{datasets} \PY{k}{import} \PY{n}{make\PYZus{}classification}
        
        \PY{k+kn}{from} \PY{n+nn}{sklearn} \PY{k}{import} \PY{n}{cross\PYZus{}validation}
        \PY{k+kn}{from} \PY{n+nn}{sklearn}\PY{n+nn}{.}\PY{n+nn}{metrics} \PY{k}{import} \PY{n}{accuracy\PYZus{}score}\PY{p}{,}\PY{n}{f1\PYZus{}score}
        \PY{k+kn}{import} \PY{n+nn}{math}
\end{Verbatim}


    \begin{Verbatim}[commandchars=\\\{\}]
{\color{incolor}In [{\color{incolor}9}]:} \PY{k+kn}{import} \PY{n+nn}{warnings}\PY{p}{;} \PY{n}{warnings}\PY{o}{.}\PY{n}{simplefilter}\PY{p}{(}\PY{l+s+s1}{\PYZsq{}}\PY{l+s+s1}{ignore}\PY{l+s+s1}{\PYZsq{}}\PY{p}{)}
\end{Verbatim}


    \subsection{Random Hill Climbing}\label{random-hill-climbing}

\subsubsection{\texorpdfstring{Show an example optimization task with
the hill climbing routine
\texttt{scipy.optimize.fmin()}}{Show an example optimization task with the hill climbing routine scipy.optimize.fmin()}}\label{show-an-example-optimization-task-with-the-hill-climbing-routine-scipy.optimize.fmin}

Hill Climbing is heuristic search used for \textbf{mathematical
optimization problems} in the field of Artificial Intelligence.

Given a large set of inputs and a good \textbf{heuristic function}, it
tries to find a sufficiently good solution to the problem. This solution
may not be the \textbf{global optimal maximum}.

\begin{itemize}
\tightlist
\item
  In the above definition, \textbf{mathematical optimization} problems
  implies that hill climbing solves the problems where we need to
  maximize or minimize a given real function by choosing values from the
  given inputs. Example-Travelling salesman problem where we need to
  minimize the distance traveled by salesman.
\item
  \textbf{Heuristic search} means that this search algorithm may not
  find the optimal solution to the problem. However, it will give a good
  solution in reasonable time.
\item
  A heuristic function is a function that will rank all the possible
  alternatives at any branching step in search algorithm based on the
  available information. It helps the algorithm to select the best route
  out of possible routes.
\end{itemize}

\begin{figure}
\centering
\includegraphics{https://i.stack.imgur.com/HISbC.png}
\caption{Random hill climbing}
\end{figure}

    \paragraph{Define a non-convex function, a pretty plotting function, and
generate an array (of say 1000
points)}\label{define-a-non-convex-function-a-pretty-plotting-function-and-generate-an-array-of-say-1000-points}

    \begin{Verbatim}[commandchars=\\\{\}]
{\color{incolor}In [{\color{incolor}10}]:} \PY{k}{def} \PY{n+nf}{func1}\PY{p}{(}\PY{n}{x}\PY{p}{)}\PY{p}{:}
             \PY{k+kn}{import} \PY{n+nn}{numpy} \PY{k}{as} \PY{n+nn}{np}
             \PY{n}{result} \PY{o}{=} \PY{p}{(}\PY{n}{x}\PY{o}{\PYZpc{}}\PY{k}{271})*np.sin(x/37)*np.exp(\PYZhy{}0.001*x)
             \PY{k}{return} \PY{n}{result}
\end{Verbatim}


    \begin{Verbatim}[commandchars=\\\{\}]
{\color{incolor}In [{\color{incolor}11}]:} \PY{k}{def} \PY{n+nf}{plot\PYZus{}fitness}\PY{p}{(}\PY{n}{array}\PY{p}{,}\PY{n}{x\PYZus{}range}\PY{o}{=}\PY{k+kc}{None}\PY{p}{)}\PY{p}{:}
             \PY{n}{fmin}\PY{o}{=}\PY{n}{array}\PY{o}{.}\PY{n}{min}\PY{p}{(}\PY{p}{)}
             \PY{n}{argfmin}\PY{o}{=}\PY{n}{array}\PY{o}{.}\PY{n}{argmin}\PY{p}{(}\PY{p}{)}
             \PY{n}{xmin}\PY{o}{=}\PY{n}{x\PYZus{}range}\PY{p}{[}\PY{n}{argfmin}\PY{p}{]}
             
             \PY{n}{plt}\PY{o}{.}\PY{n}{figure}\PY{p}{(}\PY{n}{figsize}\PY{o}{=}\PY{p}{(}\PY{l+m+mi}{15}\PY{p}{,}\PY{l+m+mi}{5}\PY{p}{)}\PY{p}{)}
             \PY{n}{plt}\PY{o}{.}\PY{n}{grid}\PY{p}{(}\PY{k+kc}{True}\PY{p}{)}
             \PY{n}{plt}\PY{o}{.}\PY{n}{title}\PY{p}{(}\PY{l+s+s2}{\PYZdq{}}\PY{l+s+s2}{Fitness function}\PY{l+s+s2}{\PYZdq{}}\PY{p}{,}\PY{n}{fontsize}\PY{o}{=}\PY{l+m+mi}{15}\PY{p}{)}
             \PY{n}{plt}\PY{o}{.}\PY{n}{xticks}\PY{p}{(}\PY{n}{fontsize}\PY{o}{=}\PY{l+m+mi}{13}\PY{p}{)}
             \PY{n}{plt}\PY{o}{.}\PY{n}{yticks}\PY{p}{(}\PY{n}{fontsize}\PY{o}{=}\PY{l+m+mi}{13}\PY{p}{)}
             \PY{n}{plt}\PY{o}{.}\PY{n}{text}\PY{p}{(}\PY{n}{x}\PY{o}{=}\PY{n}{xmin}\PY{o}{\PYZhy{}}\PY{l+m+mi}{100}\PY{p}{,}\PY{n}{y}\PY{o}{=}\PY{n}{fmin}\PY{o}{+}\PY{l+m+mi}{25}\PY{p}{,}\PY{n}{s}\PY{o}{=}\PY{l+s+s2}{\PYZdq{}}\PY{l+s+s2}{Global minima}\PY{l+s+s2}{\PYZdq{}}\PY{p}{,}\PY{n}{fontsize}\PY{o}{=}\PY{l+m+mi}{20}\PY{p}{)}
             \PY{k}{if} \PY{n+nb}{type}\PY{p}{(}\PY{n}{x\PYZus{}range}\PY{p}{)}\PY{o}{==}\PY{k+kc}{None}\PY{p}{:}
                 \PY{n}{plt}\PY{o}{.}\PY{n}{plot}\PY{p}{(}\PY{n}{array}\PY{p}{,}\PY{n}{lw}\PY{o}{=}\PY{l+m+mi}{3}\PY{p}{,}\PY{n}{c}\PY{o}{=}\PY{l+s+s1}{\PYZsq{}}\PY{l+s+s1}{blue}\PY{l+s+s1}{\PYZsq{}}\PY{p}{)}
             \PY{k}{else}\PY{p}{:}
                 \PY{n}{plt}\PY{o}{.}\PY{n}{plot}\PY{p}{(}\PY{n}{x\PYZus{}range}\PY{p}{,}\PY{n}{array}\PY{p}{,}\PY{n}{lw}\PY{o}{=}\PY{l+m+mi}{3}\PY{p}{,}\PY{n}{c}\PY{o}{=}\PY{l+s+s1}{\PYZsq{}}\PY{l+s+s1}{blue}\PY{l+s+s1}{\PYZsq{}}\PY{p}{)}
\end{Verbatim}


    \begin{Verbatim}[commandchars=\\\{\}]
{\color{incolor}In [{\color{incolor}12}]:} \PY{n}{lst}\PY{o}{=}\PY{p}{[}\PY{n}{i} \PY{k}{for} \PY{n}{i} \PY{o+ow}{in} \PY{n+nb}{range}\PY{p}{(}\PY{o}{\PYZhy{}}\PY{l+m+mi}{500}\PY{p}{,}\PY{l+m+mi}{500}\PY{p}{,}\PY{l+m+mi}{1}\PY{p}{)}\PY{p}{]}
         \PY{n}{f}\PY{o}{=}\PY{n}{np}\PY{o}{.}\PY{n}{array}\PY{p}{(}\PY{n+nb}{list}\PY{p}{(}\PY{n+nb}{map}\PY{p}{(}\PY{n}{func1}\PY{p}{,}\PY{n}{lst}\PY{p}{)}\PY{p}{)}\PY{p}{)}
\end{Verbatim}


    \begin{Verbatim}[commandchars=\\\{\}]
{\color{incolor}In [{\color{incolor}13}]:} \PY{n}{plot\PYZus{}fitness}\PY{p}{(}\PY{n}{f}\PY{p}{,}\PY{n}{np}\PY{o}{.}\PY{n}{array}\PY{p}{(}\PY{n}{lst}\PY{p}{)}\PY{p}{)}
\end{Verbatim}


    \begin{center}
    \adjustimage{max size={0.9\linewidth}{0.9\paperheight}}{output_8_0.png}
    \end{center}
    { \hspace*{\fill} \\}
    
    \paragraph{\texorpdfstring{Try hill climbing (descent) with
\texttt{x0}=220. It gets stuck at a local
minima!}{Try hill climbing (descent) with x0=220. It gets stuck at a local minima!}}\label{try-hill-climbing-descent-with-x0220.-it-gets-stuck-at-a-local-minima}

    \begin{Verbatim}[commandchars=\\\{\}]
{\color{incolor}In [{\color{incolor}25}]:} \PY{n}{res}\PY{o}{=}\PY{n}{optimize}\PY{o}{.}\PY{n}{fmin}\PY{p}{(}\PY{n}{func1}\PY{p}{,}\PY{n}{x0}\PY{o}{=}\PY{l+m+mi}{220}\PY{p}{,}\PY{n}{disp}\PY{o}{=}\PY{k+kc}{False}\PY{p}{,}\PY{n}{full\PYZus{}output}\PY{o}{=}\PY{l+m+mi}{1}\PY{p}{)}
         \PY{n+nb}{print}\PY{p}{(}\PY{n}{f}\PY{l+s+s2}{\PYZdq{}}\PY{l+s+s2}{Minima found at }\PY{l+s+si}{\PYZob{}res[0][0]\PYZcb{}}\PY{l+s+s2}{ with function value }\PY{l+s+si}{\PYZob{}res[1]\PYZcb{}}\PY{l+s+s2}{\PYZdq{}}\PY{p}{)}
         \PY{n}{plot\PYZus{}fitness}\PY{p}{(}\PY{n}{f}\PY{p}{,}\PY{n}{np}\PY{o}{.}\PY{n}{array}\PY{p}{(}\PY{n}{lst}\PY{p}{)}\PY{p}{)}
         \PY{n}{plt}\PY{o}{.}\PY{n}{vlines}\PY{p}{(}\PY{n}{x}\PY{o}{=}\PY{n}{res}\PY{p}{[}\PY{l+m+mi}{0}\PY{p}{]}\PY{p}{[}\PY{l+m+mi}{0}\PY{p}{]}\PY{p}{,}\PY{n}{lw}\PY{o}{=}\PY{l+m+mi}{3}\PY{p}{,}\PY{n}{ymin}\PY{o}{=}\PY{l+m+mi}{300}\PY{p}{,}\PY{n}{ymax}\PY{o}{=}\PY{o}{\PYZhy{}}\PY{l+m+mi}{400}\PY{p}{,}\PY{n}{color}\PY{o}{=}\PY{l+s+s1}{\PYZsq{}}\PY{l+s+s1}{k}\PY{l+s+s1}{\PYZsq{}}\PY{p}{,}\PY{n}{linestyle}\PY{o}{=}\PY{l+s+s1}{\PYZsq{}}\PY{l+s+s1}{\PYZhy{}\PYZhy{}}\PY{l+s+s1}{\PYZsq{}}\PY{p}{)}
\end{Verbatim}


    \begin{Verbatim}[commandchars=\\\{\}]
Minima found at 180.5157470703125 with function value -148.6197312132061

    \end{Verbatim}

\begin{Verbatim}[commandchars=\\\{\}]
{\color{outcolor}Out[{\color{outcolor}25}]:} <matplotlib.collections.LineCollection at 0x13ad6080>
\end{Verbatim}
            
    \begin{center}
    \adjustimage{max size={0.9\linewidth}{0.9\paperheight}}{output_10_2.png}
    \end{center}
    { \hspace*{\fill} \\}
    
    \paragraph{\texorpdfstring{Now give it a favorable starting guess
\texttt{x0}=-380. It finds the global
minima!}{Now give it a favorable starting guess x0=-380. It finds the global minima!}}\label{now-give-it-a-favorable-starting-guess-x0-380.-it-finds-the-global-minima}

    \begin{Verbatim}[commandchars=\\\{\}]
{\color{incolor}In [{\color{incolor}26}]:} \PY{n}{res}\PY{o}{=}\PY{n}{optimize}\PY{o}{.}\PY{n}{fmin}\PY{p}{(}\PY{n}{func1}\PY{p}{,}\PY{n}{x0}\PY{o}{=}\PY{o}{\PYZhy{}}\PY{l+m+mi}{380}\PY{p}{,}\PY{n}{disp}\PY{o}{=}\PY{k+kc}{False}\PY{p}{,}\PY{n}{full\PYZus{}output}\PY{o}{=}\PY{l+m+mi}{1}\PY{p}{)}
         \PY{n+nb}{print}\PY{p}{(}\PY{n}{f}\PY{l+s+s2}{\PYZdq{}}\PY{l+s+s2}{Minima found at }\PY{l+s+si}{\PYZob{}res[0][0]\PYZcb{}}\PY{l+s+s2}{ with function value }\PY{l+s+si}{\PYZob{}res[1]\PYZcb{}}\PY{l+s+s2}{\PYZdq{}}\PY{p}{)}
         \PY{n}{plot\PYZus{}fitness}\PY{p}{(}\PY{n}{f}\PY{p}{,}\PY{n}{np}\PY{o}{.}\PY{n}{array}\PY{p}{(}\PY{n}{lst}\PY{p}{)}\PY{p}{)}
         \PY{n}{plt}\PY{o}{.}\PY{n}{vlines}\PY{p}{(}\PY{n}{x}\PY{o}{=}\PY{n}{res}\PY{p}{[}\PY{l+m+mi}{0}\PY{p}{]}\PY{p}{[}\PY{l+m+mi}{0}\PY{p}{]}\PY{p}{,}\PY{n}{lw}\PY{o}{=}\PY{l+m+mi}{3}\PY{p}{,}\PY{n}{ymin}\PY{o}{=}\PY{l+m+mi}{300}\PY{p}{,}\PY{n}{ymax}\PY{o}{=}\PY{o}{\PYZhy{}}\PY{l+m+mi}{400}\PY{p}{,}\PY{n}{color}\PY{o}{=}\PY{l+s+s1}{\PYZsq{}}\PY{l+s+s1}{k}\PY{l+s+s1}{\PYZsq{}}\PY{p}{,}\PY{n}{linestyle}\PY{o}{=}\PY{l+s+s1}{\PYZsq{}}\PY{l+s+s1}{\PYZhy{}\PYZhy{}}\PY{l+s+s1}{\PYZsq{}}\PY{p}{)}
\end{Verbatim}


    \begin{Verbatim}[commandchars=\\\{\}]
Minima found at -286.6210708618164 with function value -338.18188353880555

    \end{Verbatim}

\begin{Verbatim}[commandchars=\\\{\}]
{\color{outcolor}Out[{\color{outcolor}26}]:} <matplotlib.collections.LineCollection at 0x13a41668>
\end{Verbatim}
            
    \begin{center}
    \adjustimage{max size={0.9\linewidth}{0.9\paperheight}}{output_12_2.png}
    \end{center}
    { \hspace*{\fill} \\}
    
    \subsubsection{Random restarts}\label{random-restarts}

    \begin{Verbatim}[commandchars=\\\{\}]
{\color{incolor}In [{\color{incolor}27}]:} \PY{c+c1}{\PYZsh{} Number of random restarts}
         \PY{n}{num\PYZus{}restarts}\PY{o}{=}\PY{l+m+mi}{50}
\end{Verbatim}


    \begin{Verbatim}[commandchars=\\\{\}]
{\color{incolor}In [{\color{incolor}29}]:} \PY{n}{lst\PYZus{}minima}\PY{o}{=}\PY{p}{[}\PY{p}{]}
         \PY{n}{global\PYZus{}found}\PY{o}{=}\PY{l+m+mi}{0} \PY{c+c1}{\PYZsh{}A count variable to count how many times global minima could be found by the algorithm}
         \PY{k}{for} \PY{n}{i} \PY{o+ow}{in} \PY{n+nb}{range}\PY{p}{(}\PY{n}{num\PYZus{}restarts}\PY{p}{)}\PY{p}{:}
             \PY{c+c1}{\PYZsh{} Choose a random integer between \PYZhy{}500 and 500}
             \PY{n}{init}\PY{o}{=}\PY{n}{np}\PY{o}{.}\PY{n}{random}\PY{o}{.}\PY{n}{randint}\PY{p}{(}\PY{o}{\PYZhy{}}\PY{l+m+mi}{500}\PY{p}{,}\PY{l+m+mi}{500}\PY{p}{)}
             \PY{n}{res}\PY{o}{=}\PY{n}{optimize}\PY{o}{.}\PY{n}{fmin}\PY{p}{(}\PY{n}{func1}\PY{p}{,}\PY{n}{x0}\PY{o}{=}\PY{n}{init}\PY{p}{,}\PY{n}{disp}\PY{o}{=}\PY{k+kc}{False}\PY{p}{,}\PY{n}{full\PYZus{}output}\PY{o}{=}\PY{l+m+mi}{1}\PY{p}{)}
             \PY{k}{if} \PY{n+nb}{int}\PY{p}{(}\PY{n}{res}\PY{p}{[}\PY{l+m+mi}{0}\PY{p}{]}\PY{p}{[}\PY{l+m+mi}{0}\PY{p}{]}\PY{p}{)}\PY{o}{==}\PY{o}{\PYZhy{}}\PY{l+m+mi}{286}\PY{p}{:}
                 \PY{n}{global\PYZus{}found}\PY{o}{+}\PY{o}{=}\PY{l+m+mi}{1}
             \PY{n}{lst\PYZus{}minima}\PY{o}{.}\PY{n}{append}\PY{p}{(}\PY{n}{res}\PY{p}{[}\PY{l+m+mi}{0}\PY{p}{]}\PY{p}{[}\PY{l+m+mi}{0}\PY{p}{]}\PY{p}{)}
         
         \PY{n+nb}{print}\PY{p}{(}\PY{n}{f}\PY{l+s+s2}{\PYZdq{}}\PY{l+s+s2}{With }\PY{l+s+si}{\PYZob{}num\PYZus{}restarts\PYZcb{}}\PY{l+s+s2}{ random restarts, the algorithm could find global minima }\PY{l+s+si}{\PYZob{}global\PYZus{}found\PYZcb{}}\PY{l+s+s2}{ times}\PY{l+s+s2}{\PYZdq{}}\PY{p}{)}
\end{Verbatim}


    \begin{Verbatim}[commandchars=\\\{\}]
With 50 random restarts, the algorithm could find global minima 7 times

    \end{Verbatim}

    \subsection{Simulated Annealing}\label{simulated-annealing}

\subsubsection{\texorpdfstring{Finding minima of the same fitness
function with the routine
\texttt{scipy.optimize.basinhopping()}}{Finding minima of the same fitness function with the routine scipy.optimize.basinhopping()}}\label{finding-minima-of-the-same-fitness-function-with-the-routine-scipy.optimize.basinhopping}

Simulated annealing (SA) is a \textbf{probabilistic technique} for
approximating the global optimum of a given function. Specifically, it
is a \textbf{metaheuristic to approximate global optimization in a large
search space}. It is often used when the search space is discrete (e.g.,
all tours that visit a given set of cities).

For problems where finding an approximate global optimum is more
important than finding a precise local optimum in a fixed amount of
time, simulated annealing may be preferable to alternatives such as
gradient descent.

The name and inspiration come from annealing in \textbf{metallurgy}, a
technique involving heating and controlled cooling of a material to
increase the size of its crystals and reduce their defects. Both are
attributes of the material that depend on its \textbf{thermodynamic free
energy}. Heating and cooling the material affects both the temperature
and the thermodynamic free energy.

\begin{quote}
\textbf{\emph{The simulation of annealing can be used to find an
approximation of a global minimum for a function with a large number of
variables to the statistical mechanics of equilibration (annealing) of
the mathematically equivalent artificial multiatomic system.}}
\end{quote}

This notion of slow cooling implemented in the simulated annealing
algorithm is interpreted as a slow decrease in the probability of
accepting worse solutions as the solution space is explored.
\textbf{Accepting worse solutions is a fundamental property of
metaheuristics because it allows for a more extensive search} for the
global optimal solution.

In general, the simulated annealing algorithms work as follows. At each
time step, the algorithm randomly selects a solution close to the
current one, measures its quality, and then decides to move to it or to
stay with the current solution based on either one of two probabilities
between which it chooses on the basis of the fact that the new solution
is better or worse than the current one. During the search, the
temperature is progressively decreased from an initial positive value to
zero and affects the two probabilities: at each step, the probability of
moving to a better new solution is either kept to 1 or is changed
towards a positive value; instead, the probability of moving to a worse
new solution is progressively changed towards zero.

\begin{figure}
\centering
\includegraphics{https://upload.wikimedia.org/wikipedia/commons/d/d5/Hill_Climbing_with_Simulated_Annealing.gif}
\caption{Simulated annealing}
\end{figure}

    \paragraph{\texorpdfstring{\texttt{x0}=-200, \texttt{T}=1.0,
\texttt{stepsize}=0.5. It finds a local
minima!}{x0=-200, T=1.0, stepsize=0.5. It finds a local minima!}}\label{x0-200-t1.0-stepsize0.5.-it-finds-a-local-minima}

    \begin{Verbatim}[commandchars=\\\{\}]
{\color{incolor}In [{\color{incolor}31}]:} \PY{n}{res}\PY{o}{=}\PY{n}{optimize}\PY{o}{.}\PY{n}{basinhopping}\PY{p}{(}\PY{n}{func1}\PY{p}{,}\PY{n}{x0}\PY{o}{=}\PY{o}{\PYZhy{}}\PY{l+m+mi}{200}\PY{p}{,}\PY{n}{niter}\PY{o}{=}\PY{l+m+mi}{100}\PY{p}{,}\PY{n}{T}\PY{o}{=}\PY{l+m+mf}{1.0}\PY{p}{,}\PY{n}{stepsize}\PY{o}{=}\PY{l+m+mf}{0.5}\PY{p}{)}
         \PY{n+nb}{print}\PY{p}{(}\PY{n}{f}\PY{l+s+s2}{\PYZdq{}}\PY{l+s+s2}{Minima found at }\PY{l+s+si}{\PYZob{}res.x\PYZcb{}}\PY{l+s+s2}{ with function value }\PY{l+s+si}{\PYZob{}res.fun\PYZcb{}}\PY{l+s+s2}{\PYZdq{}}\PY{p}{)}
         \PY{n}{plot\PYZus{}fitness}\PY{p}{(}\PY{n}{f}\PY{p}{,}\PY{n}{np}\PY{o}{.}\PY{n}{array}\PY{p}{(}\PY{n}{lst}\PY{p}{)}\PY{p}{)}
         \PY{n}{plt}\PY{o}{.}\PY{n}{vlines}\PY{p}{(}\PY{n}{x}\PY{o}{=}\PY{n}{res}\PY{o}{.}\PY{n}{x}\PY{p}{,}\PY{n}{lw}\PY{o}{=}\PY{l+m+mi}{3}\PY{p}{,}\PY{n}{ymin}\PY{o}{=}\PY{l+m+mi}{300}\PY{p}{,}\PY{n}{ymax}\PY{o}{=}\PY{o}{\PYZhy{}}\PY{l+m+mi}{400}\PY{p}{,}\PY{n}{color}\PY{o}{=}\PY{l+s+s1}{\PYZsq{}}\PY{l+s+s1}{k}\PY{l+s+s1}{\PYZsq{}}\PY{p}{,}\PY{n}{linestyle}\PY{o}{=}\PY{l+s+s1}{\PYZsq{}}\PY{l+s+s1}{\PYZhy{}\PYZhy{}}\PY{l+s+s1}{\PYZsq{}}\PY{p}{)}
         \PY{n}{plt}\PY{o}{.}\PY{n}{show}\PY{p}{(}\PY{p}{)}
\end{Verbatim}


    \begin{Verbatim}[commandchars=\\\{\}]
Minima found at [-251.07097382] with function value -12.337884741536309

    \end{Verbatim}

    \begin{center}
    \adjustimage{max size={0.9\linewidth}{0.9\paperheight}}{output_18_1.png}
    \end{center}
    { \hspace*{\fill} \\}
    
    \paragraph{\texorpdfstring{\texttt{x0}=-200, \texttt{T}=1.0,
\texttt{stepsize}=0.5. When we increase the number of iterations
\texttt{niter}, it hops around enough to find the global
minima!}{x0=-200, T=1.0, stepsize=0.5. When we increase the number of iterations niter, it hops around enough to find the global minima!}}\label{x0-200-t1.0-stepsize0.5.-when-we-increase-the-number-of-iterations-niter-it-hops-around-enough-to-find-the-global-minima}

    \begin{Verbatim}[commandchars=\\\{\}]
{\color{incolor}In [{\color{incolor}32}]:} \PY{n}{res}\PY{o}{=}\PY{n}{optimize}\PY{o}{.}\PY{n}{basinhopping}\PY{p}{(}\PY{n}{func1}\PY{p}{,}\PY{n}{x0}\PY{o}{=}\PY{o}{\PYZhy{}}\PY{l+m+mi}{200}\PY{p}{,}\PY{n}{niter}\PY{o}{=}\PY{l+m+mi}{2000}\PY{p}{,}\PY{n}{T}\PY{o}{=}\PY{l+m+mf}{1.0}\PY{p}{,}\PY{n}{stepsize}\PY{o}{=}\PY{l+m+mf}{0.5}\PY{p}{)}
         \PY{n+nb}{print}\PY{p}{(}\PY{n}{f}\PY{l+s+s2}{\PYZdq{}}\PY{l+s+s2}{Minima found at }\PY{l+s+si}{\PYZob{}res.x\PYZcb{}}\PY{l+s+s2}{ with function value }\PY{l+s+si}{\PYZob{}res.fun\PYZcb{}}\PY{l+s+s2}{\PYZdq{}}\PY{p}{)}
         \PY{n}{plot\PYZus{}fitness}\PY{p}{(}\PY{n}{f}\PY{p}{,}\PY{n}{np}\PY{o}{.}\PY{n}{array}\PY{p}{(}\PY{n}{lst}\PY{p}{)}\PY{p}{)}
         \PY{n}{plt}\PY{o}{.}\PY{n}{vlines}\PY{p}{(}\PY{n}{x}\PY{o}{=}\PY{n}{res}\PY{o}{.}\PY{n}{x}\PY{p}{,}\PY{n}{lw}\PY{o}{=}\PY{l+m+mi}{3}\PY{p}{,}\PY{n}{ymin}\PY{o}{=}\PY{l+m+mi}{300}\PY{p}{,}\PY{n}{ymax}\PY{o}{=}\PY{o}{\PYZhy{}}\PY{l+m+mi}{400}\PY{p}{,}\PY{n}{color}\PY{o}{=}\PY{l+s+s1}{\PYZsq{}}\PY{l+s+s1}{k}\PY{l+s+s1}{\PYZsq{}}\PY{p}{,}\PY{n}{linestyle}\PY{o}{=}\PY{l+s+s1}{\PYZsq{}}\PY{l+s+s1}{\PYZhy{}\PYZhy{}}\PY{l+s+s1}{\PYZsq{}}\PY{p}{)}
         \PY{n}{plt}\PY{o}{.}\PY{n}{show}\PY{p}{(}\PY{p}{)}
\end{Verbatim}


    \begin{Verbatim}[commandchars=\\\{\}]
Minima found at [-286.6210385] with function value -338.1818835389411

    \end{Verbatim}

    \begin{center}
    \adjustimage{max size={0.9\linewidth}{0.9\paperheight}}{output_20_1.png}
    \end{center}
    { \hspace*{\fill} \\}
    
    \paragraph{\texorpdfstring{\texttt{x0}=-100, \texttt{T}=1.0,
\texttt{stepsize}=0.5. It again finds a local
mnima.}{x0=-100, T=1.0, stepsize=0.5. It again finds a local mnima.}}\label{x0-100-t1.0-stepsize0.5.-it-again-finds-a-local-mnima.}

    \begin{Verbatim}[commandchars=\\\{\}]
{\color{incolor}In [{\color{incolor}33}]:} \PY{n}{res}\PY{o}{=}\PY{n}{optimize}\PY{o}{.}\PY{n}{basinhopping}\PY{p}{(}\PY{n}{func1}\PY{p}{,}\PY{n}{x0}\PY{o}{=}\PY{o}{\PYZhy{}}\PY{l+m+mi}{100}\PY{p}{,}\PY{n}{niter}\PY{o}{=}\PY{l+m+mi}{200}\PY{p}{,}\PY{n}{T}\PY{o}{=}\PY{l+m+mf}{1.0}\PY{p}{,}\PY{n}{stepsize}\PY{o}{=}\PY{l+m+mf}{0.5}\PY{p}{)}
         \PY{n+nb}{print}\PY{p}{(}\PY{n}{f}\PY{l+s+s2}{\PYZdq{}}\PY{l+s+s2}{Minima found at }\PY{l+s+si}{\PYZob{}res.x\PYZcb{}}\PY{l+s+s2}{ with function value }\PY{l+s+si}{\PYZob{}res.fun\PYZcb{}}\PY{l+s+s2}{\PYZdq{}}\PY{p}{)}
         \PY{n}{plot\PYZus{}fitness}\PY{p}{(}\PY{n}{f}\PY{p}{,}\PY{n}{np}\PY{o}{.}\PY{n}{array}\PY{p}{(}\PY{n}{lst}\PY{p}{)}\PY{p}{)}
         \PY{n}{plt}\PY{o}{.}\PY{n}{vlines}\PY{p}{(}\PY{n}{x}\PY{o}{=}\PY{n}{res}\PY{o}{.}\PY{n}{x}\PY{p}{,}\PY{n}{lw}\PY{o}{=}\PY{l+m+mi}{3}\PY{p}{,}\PY{n}{ymin}\PY{o}{=}\PY{l+m+mi}{300}\PY{p}{,}\PY{n}{ymax}\PY{o}{=}\PY{o}{\PYZhy{}}\PY{l+m+mi}{400}\PY{p}{,}\PY{n}{color}\PY{o}{=}\PY{l+s+s1}{\PYZsq{}}\PY{l+s+s1}{k}\PY{l+s+s1}{\PYZsq{}}\PY{p}{,}\PY{n}{linestyle}\PY{o}{=}\PY{l+s+s1}{\PYZsq{}}\PY{l+s+s1}{\PYZhy{}\PYZhy{}}\PY{l+s+s1}{\PYZsq{}}\PY{p}{)}
         \PY{n}{plt}\PY{o}{.}\PY{n}{show}\PY{p}{(}\PY{p}{)}
\end{Verbatim}


    \begin{Verbatim}[commandchars=\\\{\}]
Minima found at [-53.23064711] with function value -227.67346324906015

    \end{Verbatim}

    \begin{center}
    \adjustimage{max size={0.9\linewidth}{0.9\paperheight}}{output_22_1.png}
    \end{center}
    { \hspace*{\fill} \\}
    
    \paragraph{\texorpdfstring{\texttt{x0}=-100, \texttt{T}=1.0,
\texttt{stepsize}=0.5. What happens with more iterations? Does not seem
to
help}{x0=-100, T=1.0, stepsize=0.5. What happens with more iterations? Does not seem to help}}\label{x0-100-t1.0-stepsize0.5.-what-happens-with-more-iterations-does-not-seem-to-help}

    \begin{Verbatim}[commandchars=\\\{\}]
{\color{incolor}In [{\color{incolor}34}]:} \PY{n}{res}\PY{o}{=}\PY{n}{optimize}\PY{o}{.}\PY{n}{basinhopping}\PY{p}{(}\PY{n}{func1}\PY{p}{,}\PY{n}{x0}\PY{o}{=}\PY{o}{\PYZhy{}}\PY{l+m+mi}{100}\PY{p}{,}\PY{n}{niter}\PY{o}{=}\PY{l+m+mi}{2000}\PY{p}{,}\PY{n}{T}\PY{o}{=}\PY{l+m+mf}{1.0}\PY{p}{,}\PY{n}{stepsize}\PY{o}{=}\PY{l+m+mf}{0.5}\PY{p}{)}
         \PY{n+nb}{print}\PY{p}{(}\PY{n}{f}\PY{l+s+s2}{\PYZdq{}}\PY{l+s+s2}{Minima found at }\PY{l+s+si}{\PYZob{}res.x\PYZcb{}}\PY{l+s+s2}{ with function value }\PY{l+s+si}{\PYZob{}res.fun\PYZcb{}}\PY{l+s+s2}{\PYZdq{}}\PY{p}{)}
         \PY{n}{plot\PYZus{}fitness}\PY{p}{(}\PY{n}{f}\PY{p}{,}\PY{n}{np}\PY{o}{.}\PY{n}{array}\PY{p}{(}\PY{n}{lst}\PY{p}{)}\PY{p}{)}
         \PY{n}{plt}\PY{o}{.}\PY{n}{vlines}\PY{p}{(}\PY{n}{x}\PY{o}{=}\PY{n}{res}\PY{o}{.}\PY{n}{x}\PY{p}{,}\PY{n}{lw}\PY{o}{=}\PY{l+m+mi}{3}\PY{p}{,}\PY{n}{ymin}\PY{o}{=}\PY{l+m+mi}{300}\PY{p}{,}\PY{n}{ymax}\PY{o}{=}\PY{o}{\PYZhy{}}\PY{l+m+mi}{400}\PY{p}{,}\PY{n}{color}\PY{o}{=}\PY{l+s+s1}{\PYZsq{}}\PY{l+s+s1}{k}\PY{l+s+s1}{\PYZsq{}}\PY{p}{,}\PY{n}{linestyle}\PY{o}{=}\PY{l+s+s1}{\PYZsq{}}\PY{l+s+s1}{\PYZhy{}\PYZhy{}}\PY{l+s+s1}{\PYZsq{}}\PY{p}{)}
         \PY{n}{plt}\PY{o}{.}\PY{n}{show}\PY{p}{(}\PY{p}{)}
\end{Verbatim}


    \begin{Verbatim}[commandchars=\\\{\}]
Minima found at [-53.2306467] with function value -227.67346324906018

    \end{Verbatim}

    \begin{center}
    \adjustimage{max size={0.9\linewidth}{0.9\paperheight}}{output_24_1.png}
    \end{center}
    { \hspace*{\fill} \\}
    
    \paragraph{\texorpdfstring{\texttt{x0}=-100, \texttt{T}=10.0,
\texttt{stepsize}=0.5. What happens with higher temperature? Does not
seem to
help}{x0=-100, T=10.0, stepsize=0.5. What happens with higher temperature? Does not seem to help}}\label{x0-100-t10.0-stepsize0.5.-what-happens-with-higher-temperature-does-not-seem-to-help}

    \begin{Verbatim}[commandchars=\\\{\}]
{\color{incolor}In [{\color{incolor}35}]:} \PY{n}{res}\PY{o}{=}\PY{n}{optimize}\PY{o}{.}\PY{n}{basinhopping}\PY{p}{(}\PY{n}{func1}\PY{p}{,}\PY{n}{x0}\PY{o}{=}\PY{o}{\PYZhy{}}\PY{l+m+mi}{100}\PY{p}{,}\PY{n}{niter}\PY{o}{=}\PY{l+m+mi}{200}\PY{p}{,}\PY{n}{T}\PY{o}{=}\PY{l+m+mf}{10.0}\PY{p}{,}\PY{n}{stepsize}\PY{o}{=}\PY{l+m+mf}{0.5}\PY{p}{)}
         \PY{n+nb}{print}\PY{p}{(}\PY{n}{f}\PY{l+s+s2}{\PYZdq{}}\PY{l+s+s2}{Minima found at }\PY{l+s+si}{\PYZob{}res.x\PYZcb{}}\PY{l+s+s2}{ with function value }\PY{l+s+si}{\PYZob{}res.fun\PYZcb{}}\PY{l+s+s2}{\PYZdq{}}\PY{p}{)}
         \PY{n}{plot\PYZus{}fitness}\PY{p}{(}\PY{n}{f}\PY{p}{,}\PY{n}{np}\PY{o}{.}\PY{n}{array}\PY{p}{(}\PY{n}{lst}\PY{p}{)}\PY{p}{)}
         \PY{n}{plt}\PY{o}{.}\PY{n}{vlines}\PY{p}{(}\PY{n}{x}\PY{o}{=}\PY{n}{res}\PY{o}{.}\PY{n}{x}\PY{p}{,}\PY{n}{lw}\PY{o}{=}\PY{l+m+mi}{3}\PY{p}{,}\PY{n}{ymin}\PY{o}{=}\PY{l+m+mi}{300}\PY{p}{,}\PY{n}{ymax}\PY{o}{=}\PY{o}{\PYZhy{}}\PY{l+m+mi}{400}\PY{p}{,}\PY{n}{color}\PY{o}{=}\PY{l+s+s1}{\PYZsq{}}\PY{l+s+s1}{k}\PY{l+s+s1}{\PYZsq{}}\PY{p}{,}\PY{n}{linestyle}\PY{o}{=}\PY{l+s+s1}{\PYZsq{}}\PY{l+s+s1}{\PYZhy{}\PYZhy{}}\PY{l+s+s1}{\PYZsq{}}\PY{p}{)}
         \PY{n}{plt}\PY{o}{.}\PY{n}{show}\PY{p}{(}\PY{p}{)}
\end{Verbatim}


    \begin{Verbatim}[commandchars=\\\{\}]
Minima found at [-53.23064636] with function value -227.67346324906015

    \end{Verbatim}

    \begin{center}
    \adjustimage{max size={0.9\linewidth}{0.9\paperheight}}{output_26_1.png}
    \end{center}
    { \hspace*{\fill} \\}
    
    \paragraph{\texorpdfstring{\texttt{x0}=-100, \texttt{T}=100.0,
\texttt{stepsize}=0.5. Even higher temperature? Still not finding the
global
minima}{x0=-100, T=100.0, stepsize=0.5. Even higher temperature? Still not finding the global minima}}\label{x0-100-t100.0-stepsize0.5.-even-higher-temperature-still-not-finding-the-global-minima}

    \begin{Verbatim}[commandchars=\\\{\}]
{\color{incolor}In [{\color{incolor}36}]:} \PY{n}{res}\PY{o}{=}\PY{n}{optimize}\PY{o}{.}\PY{n}{basinhopping}\PY{p}{(}\PY{n}{func1}\PY{p}{,}\PY{n}{x0}\PY{o}{=}\PY{o}{\PYZhy{}}\PY{l+m+mi}{100}\PY{p}{,}\PY{n}{niter}\PY{o}{=}\PY{l+m+mi}{200}\PY{p}{,}\PY{n}{T}\PY{o}{=}\PY{l+m+mf}{100.0}\PY{p}{,}\PY{n}{stepsize}\PY{o}{=}\PY{l+m+mf}{0.5}\PY{p}{)}
         \PY{n+nb}{print}\PY{p}{(}\PY{n}{f}\PY{l+s+s2}{\PYZdq{}}\PY{l+s+s2}{Minima found at }\PY{l+s+si}{\PYZob{}res.x\PYZcb{}}\PY{l+s+s2}{ with function value }\PY{l+s+si}{\PYZob{}res.fun\PYZcb{}}\PY{l+s+s2}{\PYZdq{}}\PY{p}{)}
         \PY{n}{plot\PYZus{}fitness}\PY{p}{(}\PY{n}{f}\PY{p}{,}\PY{n}{np}\PY{o}{.}\PY{n}{array}\PY{p}{(}\PY{n}{lst}\PY{p}{)}\PY{p}{)}
         \PY{n}{plt}\PY{o}{.}\PY{n}{vlines}\PY{p}{(}\PY{n}{x}\PY{o}{=}\PY{n}{res}\PY{o}{.}\PY{n}{x}\PY{p}{,}\PY{n}{lw}\PY{o}{=}\PY{l+m+mi}{3}\PY{p}{,}\PY{n}{ymin}\PY{o}{=}\PY{l+m+mi}{300}\PY{p}{,}\PY{n}{ymax}\PY{o}{=}\PY{o}{\PYZhy{}}\PY{l+m+mi}{400}\PY{p}{,}\PY{n}{color}\PY{o}{=}\PY{l+s+s1}{\PYZsq{}}\PY{l+s+s1}{k}\PY{l+s+s1}{\PYZsq{}}\PY{p}{,}\PY{n}{linestyle}\PY{o}{=}\PY{l+s+s1}{\PYZsq{}}\PY{l+s+s1}{\PYZhy{}\PYZhy{}}\PY{l+s+s1}{\PYZsq{}}\PY{p}{)}
         \PY{n}{plt}\PY{o}{.}\PY{n}{show}\PY{p}{(}\PY{p}{)}
\end{Verbatim}


    \begin{Verbatim}[commandchars=\\\{\}]
Minima found at [-53.23064674] with function value -227.67346324906015

    \end{Verbatim}

    \begin{center}
    \adjustimage{max size={0.9\linewidth}{0.9\paperheight}}{output_28_1.png}
    \end{center}
    { \hspace*{\fill} \\}
    
    \paragraph{\texorpdfstring{\texttt{x0}=-100, \texttt{T}=100.0,
\texttt{stepsize}=2. Higher stepsize i.e. hopping distance? Still
hopeless
:(}{x0=-100, T=100.0, stepsize=2. Higher stepsize i.e. hopping distance? Still hopeless :(}}\label{x0-100-t100.0-stepsize2.-higher-stepsize-i.e.-hopping-distance-still-hopeless}

    \begin{Verbatim}[commandchars=\\\{\}]
{\color{incolor}In [{\color{incolor}37}]:} \PY{n}{res}\PY{o}{=}\PY{n}{optimize}\PY{o}{.}\PY{n}{basinhopping}\PY{p}{(}\PY{n}{func1}\PY{p}{,}\PY{n}{x0}\PY{o}{=}\PY{o}{\PYZhy{}}\PY{l+m+mi}{100}\PY{p}{,}\PY{n}{niter}\PY{o}{=}\PY{l+m+mi}{200}\PY{p}{,}\PY{n}{T}\PY{o}{=}\PY{l+m+mf}{100.0}\PY{p}{,}\PY{n}{stepsize}\PY{o}{=}\PY{l+m+mi}{2}\PY{p}{)}
         \PY{n+nb}{print}\PY{p}{(}\PY{n}{f}\PY{l+s+s2}{\PYZdq{}}\PY{l+s+s2}{Minima found at }\PY{l+s+si}{\PYZob{}res.x\PYZcb{}}\PY{l+s+s2}{ with function value }\PY{l+s+si}{\PYZob{}res.fun\PYZcb{}}\PY{l+s+s2}{\PYZdq{}}\PY{p}{)}
         \PY{n}{plot\PYZus{}fitness}\PY{p}{(}\PY{n}{f}\PY{p}{,}\PY{n}{np}\PY{o}{.}\PY{n}{array}\PY{p}{(}\PY{n}{lst}\PY{p}{)}\PY{p}{)}
         \PY{n}{plt}\PY{o}{.}\PY{n}{vlines}\PY{p}{(}\PY{n}{x}\PY{o}{=}\PY{n}{res}\PY{o}{.}\PY{n}{x}\PY{p}{,}\PY{n}{lw}\PY{o}{=}\PY{l+m+mi}{3}\PY{p}{,}\PY{n}{ymin}\PY{o}{=}\PY{l+m+mi}{300}\PY{p}{,}\PY{n}{ymax}\PY{o}{=}\PY{o}{\PYZhy{}}\PY{l+m+mi}{400}\PY{p}{,}\PY{n}{color}\PY{o}{=}\PY{l+s+s1}{\PYZsq{}}\PY{l+s+s1}{k}\PY{l+s+s1}{\PYZsq{}}\PY{p}{,}\PY{n}{linestyle}\PY{o}{=}\PY{l+s+s1}{\PYZsq{}}\PY{l+s+s1}{\PYZhy{}\PYZhy{}}\PY{l+s+s1}{\PYZsq{}}\PY{p}{)}
         \PY{n}{plt}\PY{o}{.}\PY{n}{show}\PY{p}{(}\PY{p}{)}
\end{Verbatim}


    \begin{Verbatim}[commandchars=\\\{\}]
Minima found at [-53.23064701] with function value -227.67346324906015

    \end{Verbatim}

    \begin{center}
    \adjustimage{max size={0.9\linewidth}{0.9\paperheight}}{output_30_1.png}
    \end{center}
    { \hspace*{\fill} \\}
    
    \paragraph{\texorpdfstring{\texttt{x0}=-100, \texttt{T}=10.0,
\texttt{stepsize}=2. But now when we increase number of iterations
\texttt{niter} with higher stepsize, we find the global
minima!}{x0=-100, T=10.0, stepsize=2. But now when we increase number of iterations niter with higher stepsize, we find the global minima!}}\label{x0-100-t10.0-stepsize2.-but-now-when-we-increase-number-of-iterations-niter-with-higher-stepsize-we-find-the-global-minima}

    \begin{Verbatim}[commandchars=\\\{\}]
{\color{incolor}In [{\color{incolor}38}]:} \PY{n}{res}\PY{o}{=}\PY{n}{optimize}\PY{o}{.}\PY{n}{basinhopping}\PY{p}{(}\PY{n}{func1}\PY{p}{,}\PY{n}{x0}\PY{o}{=}\PY{o}{\PYZhy{}}\PY{l+m+mi}{100}\PY{p}{,}\PY{n}{niter}\PY{o}{=}\PY{l+m+mi}{2000}\PY{p}{,}\PY{n}{T}\PY{o}{=}\PY{l+m+mf}{10.0}\PY{p}{,}\PY{n}{stepsize}\PY{o}{=}\PY{l+m+mi}{2}\PY{p}{)}
         \PY{n+nb}{print}\PY{p}{(}\PY{n}{f}\PY{l+s+s2}{\PYZdq{}}\PY{l+s+s2}{Minima found at }\PY{l+s+si}{\PYZob{}res.x\PYZcb{}}\PY{l+s+s2}{ with function value }\PY{l+s+si}{\PYZob{}res.fun\PYZcb{}}\PY{l+s+s2}{\PYZdq{}}\PY{p}{)}
         \PY{n}{plot\PYZus{}fitness}\PY{p}{(}\PY{n}{f}\PY{p}{,}\PY{n}{np}\PY{o}{.}\PY{n}{array}\PY{p}{(}\PY{n}{lst}\PY{p}{)}\PY{p}{)}
         \PY{n}{plt}\PY{o}{.}\PY{n}{vlines}\PY{p}{(}\PY{n}{x}\PY{o}{=}\PY{n}{res}\PY{o}{.}\PY{n}{x}\PY{p}{,}\PY{n}{lw}\PY{o}{=}\PY{l+m+mi}{3}\PY{p}{,}\PY{n}{ymin}\PY{o}{=}\PY{l+m+mi}{300}\PY{p}{,}\PY{n}{ymax}\PY{o}{=}\PY{o}{\PYZhy{}}\PY{l+m+mi}{400}\PY{p}{,}\PY{n}{color}\PY{o}{=}\PY{l+s+s1}{\PYZsq{}}\PY{l+s+s1}{k}\PY{l+s+s1}{\PYZsq{}}\PY{p}{,}\PY{n}{linestyle}\PY{o}{=}\PY{l+s+s1}{\PYZsq{}}\PY{l+s+s1}{\PYZhy{}\PYZhy{}}\PY{l+s+s1}{\PYZsq{}}\PY{p}{)}
         \PY{n}{plt}\PY{o}{.}\PY{n}{show}\PY{p}{(}\PY{p}{)}
\end{Verbatim}


    \begin{Verbatim}[commandchars=\\\{\}]
Minima found at [-286.62103857] with function value -338.18188353894107

    \end{Verbatim}

    \begin{center}
    \adjustimage{max size={0.9\linewidth}{0.9\paperheight}}{output_32_1.png}
    \end{center}
    { \hspace*{\fill} \\}
    
    \subsection{What we learned from this 1-dimensional
experiment}\label{what-we-learned-from-this-1-dimensional-experiment}

This 1-dimensional function optimization was useful to show key
properties of the randomized local search algorithms. * Both hill
climbing and simulated annealing get stuck at local minima often * We
can try random restarts for hill climb and depending on the function
complexity it will find global minima for a fraction of the time *
Simulated annealing has hyperparameters such as the anneal temperature
and step-size (hopping distance) which can be tweaked to reach global
minima even when we start from a bad position. This parameters
essentially help the algorithm \textbf{\emph{jump over}} the local hills
to reach the global minimum valley.

\subsubsection{We will apply these learning to train a neural network
next}\label{we-will-apply-these-learning-to-train-a-neural-network-next}


    % Add a bibliography block to the postdoc
    
    
    
    \end{document}
